\begin{abstractlong}
		
	Despite the recent advances in data organization and structuring, electronic medical records (EMRs) can often contain unstructured raw data, temporally constrained measurements, multichannel signal data and image data, all of which are difficult to compare and contrast in large quantities due to their sizes and variation. We present a proof of concept system that can alleviate this problem by mapping raw data to a compressed 64-dimensional space where the Euclidean distance between data points can be used as a measure of similarity. Using EEGs as a case study, we optimize a deep neural network mapping from the spectrogram of EEG data to a latent space by using the triplet loss function. To verify that this clustering method learns a meaningful representation of the data, we apply a six-class k-NN classifier to the output, a binary (seizure-like and noise-like signal) k-NN classifier to the output and visualize the output latent space using the t-SNE dimensionality reduction technique. We achieved a $60.4\%$ six-class classification accuracy, a $90.1\%$ binary classification accuracy on the TUH EEG Cohorts dataset and observed distinct clusters in a reduced dimension latent space found using the t-SNE algorithm. 		
\end{abstractlong}